% Metódy inžinierskej práce

\documentclass[10pt,slovak,a4paper]{article}

\usepackage[slovak]{babel}
%\usepackage[T1]{fontenc}
\usepackage[IL2]{fontenc} % lepšia sadzba písmena Ľ než v T1
\usepackage[utf8]{inputenc}
\usepackage{graphicx}
\usepackage{url} % príkaz \url na formátovanie URL
\usepackage{hyperref} % odkazy v texte budú aktívne (pri niektorých triedach dokumentov spôsobuje posun textu)
\usepackage{pdfpages}
\usepackage{cite}
\usepackage{dirtytalk}
%\usepackage{times}

\pagestyle{headings}

\title{Nový spôsob vzdelávania pomocou fenoménov\thanks{Semestrálny projekt v predmete Metódy inžinierskej práce, ak. rok 2020/21, vedenie: Mgr. Martin Sabo, PhD.}} % meno a priezvisko vyučujúceho na cvičeniach

\author{Rastislav Brna\\[2pt]
	{\small Slovenská technická univerzita v Bratislave}\\
	{\small Fakulta informatiky a informačných technológií}\\
	{\small \texttt{xbrna@stuba.sk}}
	}

\date{\small 30. september 2020} % upravte



\begin{document}

\maketitle

\begin{abstract}
	Vzdelávanie pomocou fenoménov je vzdelávanie v ktorom neexistujú 
	tradičné predmety ale vzdeláva sa na základe väčších tém ktoré sa rozoberajú zo všetkých strán.
	Jednu tému vieme rozobrať z fizikálneho, geografického, matematického, dejepisného alebo iného 
	hladiska. Tento štýl vzdelávania by mal zlepšit pochopenie učiva, keďze ludzký mozog vie spájať 
	si súvislosti medzi vecami omnoho lepšie ako si pamätať fakty naspamäť. V niektorých krajinách ktoré
	patria medzi lídrov vo vzdelávaní sa takýto systém už pomaly začína používať v praxi. Štúdia z turecka,
	potvrdila zlepšenie priemeru študentov o viac ako 10\%. Taktiež tento typ vzdelávania pomohol študentom 
	si dlhšie zachovať znalosťi ktoré nadobudly.\cite{jcer553507} 
\end{abstract}

\section{Úvod}

Svet a technológie idú dopredu ale sposob vzdelávanie je zastaralý a stovky rokov rovnaký. Znalosti potrebné k
životu sa menia a preto sa potrebuje prispôsobiť aj vzdelávanie. Aktuálny vzdelávací systém vo väčšine krajín 
rozdeluje vzdelávanie na určité smery(predmety), núti študentov sa učit vaľa veci z pamäti, vyvíja zbytočný tlak
a nezmyselne stresuje študentov už od útleho veku. Vzdelávanie na základe
fenoménov (Phenomenon-Based Learning) je metódou vzdelávania ktorá by mala zvýšit efektivitu vzdelávania
a dodať študentom lepšiu zručnosť, kreativitu, kritické myslenie, a schopnosť koloaborovať. Závisí na študovaní
fenoménu reálneho sveta z rôznych smerov pomocou čoho prepája hranice medzi školskými predmetami ako ich poznáme.
Mení vysvetlovanie novej látky z vysvetlenia učiva, vhoršom prípade len napísania poznámok z látky na tabulu na úvádzanie
študentou do témy cez zauhjímavé príbehy alebo hry a nabádanie ich k otázkam vďaka ktorým si znalosť témy osvoja a hlbšie zapametajú.
Aktívne núti študentov kolaborovať medzi sebov a zúčastnovať sa na spoločných aktivitách za účelom riešenia
problémov a odpovedania na otázky. 

\section{Definícia pojmov}

\subsection{Fenomén}

Fenomén môžeme definovať ako niečo čo existuje a môžeme to vidieť, cítiť, ochutnat, atd., hlane niečo neobyčajné
a zaujímavé. Príkladom pre fenomén môžeu byť vedecké objavy, trendy v technológiách alebo prírodné katastrofy.\cite{pblf}

%\subsection{Vzdelávanie na základe fenoménov}

%Vzdelávanie na základe fenoménov tzv.\say{Phenomenon-Based Learning} alebo skrátene PhenoBL. 

\section{Jadro}



\section{Záver}

Výsledký o vzdelávaniana základe fenoménov nachádzajú zlepšenie vo výsledkoch študentov. Fínsko po implementácii
tohoto vzdelávacieho systému priamo do školstva v celej krajine sa stále drží medzi top krajinami na svete a zaznamenalo 
nárast úspešnosti študentov. Systém sa javý ako veľkým zlepšením oproti predchádzajúcemu avšak úpravu budú potrebujú aj
štandardizované testy ktoré su priamo viazané na naspameť fakty na ktoré nieje kladený z dobrých dôvodov dôraz.

\bibliography{literature}
\bibliographystyle{apalike}

\end{document}
